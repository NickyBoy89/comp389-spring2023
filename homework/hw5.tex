\documentclass{article}

\usepackage{amsmath}
\usepackage{amsthm}
\usepackage{amssymb}
\usepackage{amsfonts}
\usepackage[a4paper, total={6in, 8in}]{geometry}

\DeclareMathOperator{\Enc}{\textsf{Enc}}

\newtheorem{analysis}{Analysis}

\title{Homework 5}
\date{4 May 2023}
\author{Nicholas Novak (), Yash Karandikar (),\\
Elise Corwin (), Jadriel Delim ()}

\begin{document}
\maketitle

\begin{enumerate}
  \item Consider the following key-exchange protocol:
    \begin{enumerate}
      \item Alice chooses uniform $k, r \in \left\{0, 1\right\}^n$, and sends $s
        := k \oplus r$ to Bob.
      \item Bob chooses uniform $t \in \left\{0, 1\right\}^n$, and sends $u :=
        s \oplus t$ to Alice.
      \item Alice computes $w := u \oplus r$ and sends $w$ to Bob.
      \item Alice outputs $k$ and Bob outputs $w \oplus t$.
    \end{enumerate}

    Show that Alice and Bob output the same key. Analyze the security of
    scheme (i.e. either prove its security or show a concrete attack).

    \begin{analysis}
      Consider the following key exchange. Alice chooses uniform $k, r \in
      \left\{0, 1\right\}^n$ and sends Bob $k \oplus r$. Bob chooses uniform $t
      \in \left\{0, 1\right\}^n$ and sends $k \oplus r \oplus t$ back. Then
      Alice computes $k \oplus r \oplus t \oplus r$, which is $k \oplus t$, and
      sends it back to Bob. Alice outputs $k$ and Bob outputs $k \oplus t \oplus
      t$ which is $k$. So, they output the same key.
    \end{analysis}

    \begin{proof}
      Consider a replay attack. Let $\mathcal{A}$ be a PPT adversary who
      intercepts Alice's initial message and replays it (such that Bob receives
      it twice). Then, Bob receives $k \oplus r \oplus k \oplus r$ where $k, r
      \in \left\{0, 1\right\}^n$. Bob chooses uniform $t \in \left\{0,
      1\right\}^n$ and sends back $k \oplus r \oplus k \oplus r \oplus t$.

      Alice then sends $k \oplus r \oplus k \oplus r \oplus t \oplus r$, which
      is $r \oplus t$. Bob outputs $r \oplus t \oplus t$, which is $r$, but
      Alice outputs $k$. Therefore, the scheme is broken.
    \end{proof}

  \item Consider the following invariant of El Gamal encryption. Let $p = 2q +
    1$, let $\mathbb{G}$ be the group of squares modulo $p$ (so $\mathbb{G}$
    is a subgroup of $\mathbb{Z}^{*}_p$ of order $q$), and let $g$ be a
    generator of $\mathbb{G}$. The private key is $(\mathbb{G}, g, q, x)$ and
    the public key is $(\mathbb{G}, g, q, h)$, where $h = g^x$ and $x \in
    \mathbb{Z}_q$ is chosen uniformly. To encrypt a message $m \in
    \mathbb{Z}_q$, choose a uniform $r \in \mathbb{Z}_q$, compute $c_1 := g^r
    \mod p$ and $c_2 := h^r + m \mod p$, and let the ciphertext be $\left<c_1,
    c_2\right>$. Is this scheme CPA-secure? Prove your answer.

    \begin{proof}
      Let $p, q \in \mathbb{Z}^{+}$ such that $p, q$ are distinct and $p = 2q +
      1$. Let $\mathcal{G}$ be the subgroup of $\mathbb{Z}_p^{*}$ of order $q$.
      Let $m, r \in \mathbb{Z}_q$. Compute $c_1 := g^r \mod{p}$ and $c_2 := h_r
      + m \mod p$. Suppose $h^r + m_1 \mod p \in \mathbb{Z}_p^{*}$. For the
      scheme to be CPA-secure, two messages $m_1, m_2 \in \mathbb{Z}_q$ and
      their $\delta$ counterparts $(m_b, \delta)$ need to be indistinguishable.

      Since we supposed $h^r + m_1 \mod p \in \mathbb{Z}_p^{*}$, show that $h^r
      - m$, so the message $\delta$ pairs are distinguishable. Now, since
      $\mathcal{G}$ is a subgroup of $\mathbb{Z}_p^{*}$, $(h^r + m_1) \mod p
      \in \mathbb{Z}_p^{*}$ (Since the group is of order $q$). So, for distinct
      $m_1, m_2 \in \mathbb{Z}_p^{*}$, we have the same $\delta$. That means
      that the messages are not indistinguishable, so the scheme is not
      CPA-secure.
    \end{proof}

  \item Consider the RSA encryption function $E(x) = x^e \mod n$ where $n = pq$
    with $p$ and $q$ being two distinct prime numbers, and $\gcd(e, (p - 1), (q
    - 1)) = 1$. Let $D(x)$ be the corresponding decryption function such that
    $D(E(x)) = x$ for all $x \in \mathbb{Z}^{*}_n$. Suppose $eh = 1 \mod p - 1$.
    Show that $D(x) = x^h \mod p$.

    \begin{proof}
      Let $p, q \in \mathbb{Z}^{+}$ be distinct primes. Let $n \in
      \mathbb{Z}^{+}$ such that $n = pq$.

      Let $e, h \in \mathbb{Z}^{+}$ be inverses ($\mod (p - 1)$) such that $e,
      n$ are coprime.

      Suppose $E(x) = x^e \mod{n}$, and suppose $eh \equiv 1 \pmod{p - 1}$.
      Then, suppose $D(E(x)) = x$.

      Then $D(x^e \mod n) = x$.

      Consider $x^h$. Since $p \mid pq, x^h \mod p \equiv x^h \mod{pq}$. Since
      $n = pq$, $D(x^e \mod{pq}) = x$.

      We supposed that $eh \equiv 1 \pmod{p - 1}$. Since $p - 1 \mid (p - 1)(q -
      1)$, and $p \mid pq$, $D(x^e \mod{pq}) = D(x^e \mod{p}) = x$.

      Since $e$ and $h$ are inverses. $D(x^e \mod{p}) = e^{en} \mod{p} = x$. So,
      $D(x) = x^h \mod{p}$.
    \end{proof}
  \item
    \begin{enumerate}
      \item In class we showed an attack on the plain RSA signature scheme in
        which an attacker forges a signature on an arbitrary message using two
        different signing queries. Show how an attacker can forge a signature on
        an arbitrary message using a single signing query.

        \begin{proof}
          Let $p, q \in \mathbb{Z}^{+}$ be distinct primes. Let $N \in
          \mathbb{Z}$ such that $N = pq$. Let $e, d \in \mathbb{Z}^{+}$ such
          that $ed \equiv 1 \pmod{\phi(N)}$. Let $m \in \mathbb{Z}^{*}_n$.
          Suppose an attacker $\mathbb{A}$ has an arbitrary message $m$.
          Consider $\sigma$ such that $\sigma = \left[m \mod N\right]$.

          This is one signing query. Now, construct $\sigma'$ such that $\sigma'
          = \sigma * \sigma = \left[m * m \mod N\right]$. With exactly one
          signing query, we have forged a signature.
        \end{proof}
      \item Assume the RSA problem is hard. Show that the plain RSA signature
        scheme satisfies the following weak definition of security: an attacker
        is given the public key $\left<N, e\right>$ and a uniform message $m \in
        \mathbb{Z}^{*}_N$. The adversary succeeds if it can output a valid
        signature on $m$ without making any signing queries.

        \begin{proof}
          Suppose that the RSA problem is hard. Then RSA cannot be solved in
          polynomial time. Consider an experiment in which an attacker
          $\mathcal{A}$ is given a public key $\left<N, e\right>$ and uniform $m
          \in \mathbb{Z}^{*}_N$. Since RSA cannot be solved in polynomial time,
          $\mathcal{A}$ must make some queries to the
          $\textsf{Sign}_{sk}(\cdot)$ oracle to create a $(m, \gamma)$ pair to
          forge. This information is provided to the oracle, implies that $m \in
          \mathcal{M}$ and $\mathcal{A}$ cannot provide $m$ as output in the
          experiment. Furthermore, $\mathcal{A}$ cannot bear RSA by chance
          (since we supposed RSA is hard). Since $\mathcal{A}$ cannot succeed,
          the plain RSA scheme satisfies the weak definition of security.
        \end{proof}
    \end{enumerate}
  \item Consider a variant of DSA as follows: $p$ and $q$ are prime where $q$
    divides $p - 1$, and $\alpha \in \mathbb{Z}^{*}_p$ is of order $q$. Alice chooses
    some secret $a$ as the private signing key and the public verification key
    is $\beta = \alpha^a \mod p$. The public parameters are $p, q, \alpha$ and a
    collision resistant hash function $h$. To sign a message $x$, Alice computes
    $y = h(x)$ and generates a random secret $k, 1 \leq k \leq q - 1$. Then
    signature is $\left<r, s\right>$ where $r = (\alpha^k \mod p) \mod q$, and
    $s = k - rya \mod q$.
    \begin{enumerate}
      \item Describe how a signature $\left<r, s\right>$ on a message $x$ can be
        publicly verified.

        \begin{proof}
          Publicly verify $\left<r, s\right>$ on message $x$. Let $k \in
          \mathbb{Z}^{+}$ and let $\alpha \in \mathbb{Z}_p^{*}$ be a random
          distinct secret. Let $p, q \in \mathbb{Z}^{+}$ be distinct primes.

          Since $r = (\alpha^k \pmod{p}) \pmod {q}$, we can use $\alpha^k$ to
          find $s$. Since $s = k - rya \pmod{q}$, find $k$ to solve for
          $\alpha^k$. By $s = k - rya, k = s + rya$.

          Then
          \begin{align*}
            (\alpha^k \mod{p}) \mod q &= (\alpha^k \mod{p}) \mod{q}\\
                                &= (\alpha^s \alpha^{rya} \mod{p}) \mod{q}\\
                                &= (\alpha^k \alpha^{ry} \alpha^a \mod{p}) \mod{q}\\
                                &= \beta (\alpha^s \alpha^{ry} \alpha^a \mod{p}) \mod{q}
          \end{align*}

          So, $r = (\beta \alpha^s \alpha^{ry} \mod{p}) \mod{q}$ exactly when
          $\left<r, s\right> = h(x)$, thereby verifying the message.
        \end{proof}

      \item Is it secure to use the same $k$ to sign two different messages and
        why?

        \begin{proof}
          Suppose the same $k$ is used to sign two different messages. Then by
          definition, $s_1 = k - ry_1a$ and $s_2 = k - ry_2a$.

          Subtract the second equation from the first to get $s_1 = s_2 = -ry_1a
          + ry_2a = a(ry_2 - ry_1)$.

          So, $a = \frac{s_1 - s_2}{r(y_2 - y_2)}$. Since $r = (\alpha^k
          \mod{p}) \mod{q}$, the same secret key $k$ is used for the $\delta$
          generation of the private signing key $a$ in both equations. The
          definition of a successful attack on $\textsf{Sign}_{sk}(m, \delta)$
          is to forge some $(m, \delta)$ pair that works such that $m \notin M$.
          Here, we have the same $(m, \delta)$ pairs for our different messages
          and we didn't query an oracle. So, using the same key is insecure.
        \end{proof}

    \end{enumerate}
\end{enumerate}

\end{document}
