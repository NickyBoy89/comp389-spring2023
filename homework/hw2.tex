\documentclass{article}

\usepackage{amsmath}
\usepackage{amsfonts}
\usepackage[a4paper, total={6in, 8in}]{geometry}

\newtheorem{definition}{Definition}

\title{Homework 2}
\date{7 March 2023}
\author{Nicholas Novak (), Yash Karandikar (),
Elise Corwin (), Jadriel Delim ()}

\begin{document}
\maketitle

\begin{enumerate}
  \item Let $F$ be a keyed function and consider the following experiment:

    \textbf{The PRF indistinguishability experiment} $\text{PRF}_{\mathcal{A},
    F}(n)$:

    \begin{enumerate}
      \item[(i)] A uniform bit $b \in \left\{0, 1\right\}$ is chosen. If $b = 1$ then
        choose uniform $k \in \left\{0, 1\right\}^n$.
      \item[(ii)] $\mathcal{A}$ is given $1^n$ for input. If $b = 0$ then
        $\mathcal{A}$ is given access to a uniform function $f \in
        \text{Func}_n$. If $b = 1$ then $\mathcal{A}$ is instead given access to
        $F_k(\cdot)$.
      \item[(iii)] $\mathcal{A}$ outputs a bit $b'$.
      \item[(iv)] The output of the experiment is defined to be 1 if $b' = b$, and $0$
        otherwise.
    \end{enumerate}

    Define pseudorandom functions using this experiment, and prove that your
    definition is equivalent to the following definition we've seen in class.

    \begin{definition}
      Let $F : \left\{0, 1\right\}^* \times \left\{0, 1\right\}^* \to \left\{0,
      1\right\}^*$ \textit{be an efficient, length-preserving keyed function.}
      $F$ \textit{is \textbf{pseudorandom function} if for all probabilistic
        polynomial-time distinguishers D, there is a negligible function
      \textbf{negl} such that:}

      \[
        \left| \Pr\left[D^{F_k(\cdot)}(1^n) = 1\right] -
      \Pr\left[D^{f(\cdot)}(1^n) = 1\right] \right| \leq \text{negl}(n),
      \]

      \textit{where the first probability is taken over uniform choice of $k \in
        \left\{0, 1\right\}^n$ and the randomness of $D$, and the second
        probability is taken over uniform choice of $f \in \text{Func}_n$ and
      randomness of $D$.}
    \end{definition}
  \item Define a notion of perfect secrecy under a chosen-plaintext attack by
    adapting the following definition we've seen in class.

    \begin{definition}
      \textit{A private-key encryption scheme $\Pi = (\textbf{Gen}, \textbf{Enc},
        \textbf{Dec})$ has \textbf{indistinguishable encryptions under a
        chosen-plaintext attack}, or is \textbf{CPA-secure}, if for all
        probailistic polynomial time adversaries $\mathcal{A}$ there is a
      negligible function \textbf{negl} such that}

        \[
          \Pr\left[\text{PrivK}_{\mathcal{A}, \Pi}^\text{cpa}(n) = 1\right] \leq
          \frac{1}{2} + \textbf{negl}(n),
        \]

        \textit{where the probability is taken over the randomness used by
        $\mathcal{A}$, as well as the randomness used in the experiment.}

        \textit{Show that the definition cannot be achieved.}
    \end{definition}
  \item Let $F$ be a pseudorandom permutation, and define a fixed-length
    encryption scheme $(\text{Enc}, \text{Dec})$ as follows: On input $m \in
    \left\{0, 1\right\}^{n / 2}$ and key $k \in \left\{0, 1\right\}^n$,
    algorithm \textbf{Enc} chooses a uniform string $r \in \left\{0,
    1\right\}^{n/2}$ of length $n/2$ and computes $c := F_k(r \Vert m)$.

    Show how to decrypt, and prove that this scheme is CPA-secure for
    messages of length $n / 2$.
  \item Let $F$ be a pseudorandom function and $G$ be a pseudorandom generator
    with expansion factor $l(n) = n + 1$. For each of the following encryption
    schemes, state whether the scheme has indistinguishable encryptions in the
    presence of an eavesdropper and whether it is CPA-secure.

    (In each case, the shared key is a uniform $k \in \left\{0, 1\right\}^n$.)
    Explain your answer.

    \begin{enumerate}
      \item To encrypt $m \in \left\{0, 1\right\}^{n + 1}$, choose uniform $r
        \in \left\{0, 1\right\}^n$ and output the ciphertext $\left\langle r,
          G(r) \oplus m\right\rangle$.
        \item To encrypt $m \in \left\{0, 1\right\}^n$, output the ciphertext $m
          \oplus F_k(0^n)$.
        \item To encrypt $m \in \left\{0, 1\right\}^{2n}$, parse $m$ as $m_1
          \Vert m_2$ with $| m_1 | = | m_2 |$, then choose uniform
          $r \in \left\{0, 1\right\}^n$ and send $\left\langle r, m_1 \oplus
          F_k(r), m_2 \oplus F_k(r + 1)\right\rangle$.
    \end{enumerate}
  \item Show that the CBC and CTR modes of operation do not yield CCA-secure
    encryption schemes (regardless of $F$).
\end{enumerate}

\end{document}
