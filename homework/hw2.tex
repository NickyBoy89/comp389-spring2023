\documentclass{article}

\usepackage{amsthm}
\usepackage{amsmath}
\usepackage{amsfonts}
\usepackage[a4paper, total={6in, 8in}]{geometry}

\newtheorem{definition}{Definition}

\DeclareMathOperator{\PRF}{\textsf{PRF}}
\DeclareMathOperator{\Func}{\textsf{Func}}
\DeclareMathOperator{\negl}{\textsf{negl}}
\DeclareMathOperator{\PrivK}{\textsf{PrivK}}
\DeclareMathOperator{\Gen}{\textsf{Gen}}
\DeclareMathOperator{\Enc}{\textsf{Enc}}
\DeclareMathOperator{\Dec}{\textsf{Dec}}
\DeclareMathOperator{\IV}{\textsf{IV}}

\title{Homework 2}
\date{7 March 2023}
\author{Nicholas Novak (), Yash Karandikar (),
Elise Corwin (), Jadriel Delim ()}

\begin{document}
\maketitle

\begin{enumerate}
  \item Let $F$ be a keyed function and consider the following experiment:

    \textbf{The PRF indistinguishability experiment} $\PRF_{\mathcal{A},
    F}(n)$:

    \begin{enumerate}
      \item[(i)] A uniform bit $b \in \left\{0, 1\right\}$ is chosen. If $b = 1$ then
        choose uniform $k \in \left\{0, 1\right\}^n$.
      \item[(ii)] $\mathcal{A}$ is given $1^n$ for input. If $b = 0$ then
        $\mathcal{A}$ is given access to a uniform function $f \in
        \Func_n$. If $b = 1$ then $\mathcal{A}$ is instead given access to
        $F_k(\cdot)$.
      \item[(iii)] $\mathcal{A}$ outputs a bit $b'$.
      \item[(iv)] The output of the experiment is defined to be 1 if $b' = b$, and $0$
        otherwise.
    \end{enumerate}

    Define pseudorandom functions using this experiment, and prove that your
    definition is equivalent to the following definition we've seen in class.

    \begin{definition}
      Let $F : \left\{0, 1\right\}^* \times \left\{0, 1\right\}^* \to \left\{0,
      1\right\}^*$ \textit{be an efficient, length-preserving keyed function.}
      $F$ \textit{is \textsf{pseudorandom function} if for all probabilistic
        polynomial-time distinguishers D, there is a negligible function
      \textsf{negl} such that:}

      \[
        \left| \Pr\left[D^{F_k(\cdot)}(1^n) = 1\right] -
      \Pr\left[D^{f(\cdot)}(1^n) = 1\right] \right| \leq \textsf{negl}(n),
      \]

      \textit{where the first probability is taken over uniform choice of $k \in
        \left\{0, 1\right\}^n$ and the randomness of $D$, and the second
        probability is taken over uniform choice of $f \in \Func_n$ and
      randomness of $D$.}
    \end{definition}

    Pseudorandom according to this experiment is when the adversary cannot determine 
    whether the output comes from a pseudorandom function or a uniform function. And according
    to this problem, the output is defined to be 1 or 0 indicating that you cannot distinguish between the outputs. 
    
    \begin{proof}
      Choose a uniform bit $b \in \left\{0, 1\right\}$. If $b = 1$, then
      generate a uniform key $k \in \left\{0, 1\right\}^n$.

      Fix a PPT attacker $\mathcal{A}$ that receives $1^n$ as an input. If
      \begin{enumerate}
        \item $b = 0$, then give $\mathcal{A}$ a uniform function $f \in
          \Func_n$.
        \item $b = 1$, then give $\mathcal{A}$ a keyed pseudorandom function
          $F_k$.
      \end{enumerate}

      The attacker $\mathcal{A}$ then runs the PRF indistinguishability
      experiment, and outputs a new bit $b'$. If $b = b'$, then the experiment
      is considered a success.

      Create a distinguisher $D$ that uses $\mathcal{A}$ as a subroutine, and
      returns whether the attacker could successfully distinguish between the
      two schemes.

      \begin{itemize}
        \item If $b = 0$, $\mathcal{A}$ is given a uniform function $f \in \Func_n$.

          Due to the definition of uniform functions, the chance that
          $D$ successfully distinguishes two messages is exactly

          \[
            \Pr \left[D^{f(\cdot)}(1^n) = 1\right] = \frac{1}{2}
          \].

        \item If $b = 1$, $\mathcal{A}$, is given a pseudorandom keyed function
          $F_k$.

          Due to the definition of pseudorandom functions, the chance that
          $D$ can distinguish between two messages is
          $\frac{1}{2}$, plus some negligible amount $\negl(n)$. i.e.

          \[
            \Pr \left[D^{F_k(\cdot)} = 1\right] \leq \frac{1}{2} + \negl(n)
          \]
      \end{itemize}

      For this experiment, we need to show that the attacker $\mathcal{A}$
      cannot distinguish between what encryption function it receives as well.
      We show this as

      \begin{align*}
        \left| \Pr \left[D^{f(\circ)}(1^n) = 1\right] - \Pr
        \left[D^{F_k(\circ)}(1^n) - 1\right]\right| &\leq \left| \frac{1}{2} -
        \frac{1}{2} + \negl(n) \right|\\
                                                    &\leq \negl(n)
      \end{align*}

      Thus, we find an equivalent definition that $\mathcal{A}$ can only
      distinguish between the two encryption schemes with negligible
      probability.
    \end{proof}
  \item Define a notion of perfect secrecy under a chosen-plaintext attack by
    adapting the following definition we've seen in class.

    \begin{definition}
      \textit{A private-key encryption scheme $\Pi = (\Gen, \Enc,
        \Dec)$ has \textsf{indistinguishable encryptions under a
        chosen-plaintext attack}, or is \textsf{CPA-secure}, if for all
        probailistic polynomial time adversaries $\mathcal{A}$ there is a
      negligible function \textsf{negl} such that}

        \[
          \Pr\left[\PrivK_{\mathcal{A}, \Pi}^\text{cpa}(n) = 1\right] \leq
          \frac{1}{2} + \textsf{negl}(n),
        \]

        \textit{where the probability is taken over the randomness used by
        $\mathcal{A}$, as well as the randomness used in the experiment.}

        \textit{Show that the definition cannot be achieved.}
    \end{definition}

    Perfect secrecy under a chosen-plaintext attack would mean that the 
    adversary $\mathcal{A}$ chooses $m_0$ and $m_1$, knows the challenge ciphertext
    $c_0$ and $c_1$, and has access to the encryption oracle.

    We cannot meet this definition of perfect secrecy. Adversary $\mathcal{A}$ is given a challenge ciphertext c as part of the experiment. By chosen-plaintext attack, $\mathcal{A}$ has access to ciphertext $c_0$ and $c_1$:

    \begin{equation}
        c_0 = Enc_k(m_0)
    \end{equation}

    \begin{equation}
        c_1 = Enc_k(m_1)
    \end{equation}

    Construct a distinguisher D to determine if $c_0$ or $c_1$ is the pseudorandomly generated ciphertext. Exactly one of them will be generated from the pseudorandom function F. \\
    
    Now, the distinguisher D captures the pseudorandomly generated ciphertext with probability 1. D also captures the truly uniform ciphertext with probability $\frac{1}{2^n}$. So,

    \[
        \left| \Pr\left[D^{F_k(\cdot)}(1^n) = 1\right] -
      \Pr\left[D^{f(\cdot)}(1^n) = 1\right] \right| 
      = 1 - \frac{1}{2^n}
      \geq 1/2,
      \]
    
  \item Let $F$ be a pseudorandom permutation, and define a fixed-length
    encryption scheme $(\Enc, \Dec)$ as follows: On input $m \in
    \left\{0, 1\right\}^{n / 2}$ and key $k \in \left\{0, 1\right\}^n$,
    algorithm $\Enc$ chooses a uniform string $r \in \left\{0,
    1\right\}^{n/2}$ of length $n/2$ and computes $c := F_k(r \Vert m)$.

    How to decrypt:

    \begin{align*}
      c &= F_k(r \Vert m)\\
    \end{align*}
    \begin{align*}
      r \Vert m &= F^{-1}_k(c)\\
    \end{align*}

    Now, we can get $m$ by cutting off the first $n / 2$ bits of $r \Vert m$.

    Show how to decrypt, and prove that this scheme is CPA-secure for
    messages of length $n / 2$.
  \item Let $F$ be a pseudorandom function and $G$ be a pseudorandom generator
    with expansion factor $l(n) = n + 1$. For each of the following encryption
    schemes, state whether the scheme has indistinguishable encryptions in the
    presence of an eavesdropper and whether it is CPA-secure.

    (In each case, the shared key is a uniform $k \in \left\{0, 1\right\}^n$.)
    Explain your answer.

    % https://crypto.stackexchange.com/questions/31309/given-a-pseudorandom-function-f-is-f-kr-f-kr-1-also-a-prf?rq=1

    \begin{enumerate}
      \item To encrypt $m \in \left\{0, 1\right\}^{n + 1}$, choose uniform $r
        \in \left\{0, 1\right\}^n$ and output the ciphertext $\left\langle r,
          G(r) \oplus m\right\rangle$.

          \begin{proof}
            $G$ does not use a key that changes between functions, it is
            deterministic.

            Since the attacker has access to $G$, and can find the $r$ via the
            first $n / 2$ bits of the ciphertext, we can decrypt the function
            with a $\Pr = 1$.

            Therefore, it is neither EAV-secure, nor CPA-secure.
          \end{proof}

        \item To encrypt $m \in \left\{0, 1\right\}^n$, output the ciphertext $m
          \oplus F_k(0^n)$.

          \begin{proof}
            In the CPA-security experiment, we have access to some decryption
            oracle for $K_k$.

            However, since the value to $K$ is just a unary string of length
            $n$, once we observe the output for $F_k(0^n)$ for one message of
            length $n$, we can query the decryption oracle with the same unary
            string of length $n$, and XOR that with the ciphertext to get the
            message. Therefore, this scheme is not CPA-secure.

            However, for EAV-security, consider the case where the pseudorandom
            function $F_k$ is replaced by a random function $f \in \Func_n$.

            If this is the case, we get some random pad.

            Create some distinguisher $D$:

            Choose two messages $m_0, m_1 \in \left\{0, 1\right\}^n$. Now, we
            will see if $D$ can distinguish between them with non-negligible
            probability, which will contradict our assumption that $F_k$ is a
            random function.

            Choose some uniform bit $b \in \left\{0, 1\right\}$.

            Use the encryption oracle $\mathcal{O}$ to encrypt the message
            $m_b$. If $\mathcal{O} = f$ is a random function, the probability of
            success will be $\frac{1}{2}$. However, if $\mathcal{O} = F_k$, then
            due to the definition of pseudorandom, $D$ will succeed with a
            probability of $\frac{1}{2} + \negl$, which does not contradict our
            initial assumption.

            Therefore, the scheme is EAV-secure, but not CPA-secure.
          \end{proof}

        \item To encrypt $m \in \left\{0, 1\right\}^{2n}$, parse $m$ as $m_1
          \Vert m_2$ with $| m_1 | = | m_2 |$, then choose uniform
          $r \in \left\{0, 1\right\}^n$ and send $\left\langle r, m_1 \oplus
          F_k(r), m_2 \oplus F_k(r + 1)\right\rangle$.

          % Proof Idea:
          % Use a random function f instead of a F_k
          % if the distinguisher can tell the difference, then it is not
          % CPA-secure
          %
          % Also, if something is CPA-secure, that is stronger than EAV, so
          % anything CPA-secure is also EAV secure

          \begin{proof}
            Within every message, the scheme is CPA-secure because it uses the
            same method as the CTR security mode.

            Since every message is CPA-secure, over multiple messages, the
            scheme is CPA-secure, which is a stronger guarantee than
            EAV-security.
          \end{proof}
    \end{enumerate}
  \item Show that the CBC and CTR modes of operation do not yield CCA-secure
    encryption schemes (regardless of $F$).

    % To prove this, create an experiment where you distinguish between two
    % messages that you choose, and two vectors of <IV, seed>

    \begin{proof}
      Start with a scheme using the CTR mode of operation $\Pi$.

      We have en encryption oracle $\Enc(\cdot)$ and a decryption oracle
      $\Dec(\cdot)$ that will decrypt every ciphertext, except for the challenge
      ciphertext.

      Fix two messages $m_0 = 0^n, m_1 = 1^n$, and choose a uniform bit $b \in
      \left\{0, 1\right\}$.

      Use the encryption oracle $\Enc$ to encrypt a random message $m_b$, and
      get the challenge ciphertext $c = F_k(\IV \Vert i) \oplus m_b$, where $i$
      is the sequence number in CTR encryption.

      Flip the first bit of $c$, changing it to $c'$, and request decryption of
      that ciphertext from the decryption oracle $\Dec(c')$, which will output
      some message $m'$.

      If the bits of $m'$ are $0 1^{n - 1}$, then output the bit $b' = 1$.
      Otherwise, output the bit $b' = 0$. Since this experiment will succeed
      with $\Pr = 1$, we find that the CTR-mode of operation is not CCA-secure.
    \end{proof}

    % https://crypto.stackexchange.com/questions/62564/is-aes-cbc-mode-not-secure-against-chosen-cipher-text-attacks-even-if-the-iv-is

    \begin{proof}
      Start with a scheme using the CBC method of operation $\Pi$.

      Fix two message $m_0 = 0^n, m_1 = 1^n$, and choose a uniform bit $b \in
      \left\{0, 1\right\}$.

      Use the encryption oracle to encrypt a random message $m_b$, and get the
      challenge ciphertext $c = \left\langle c_{i - 1}, F_k(c_{i - 1} \oplus
      m_b)\right\rangle $ where $c_{i - 1}$ is
      the previous message in the chain, or the $\IV$.

      Flip the first bit of the previous message $c_{i - 1}$, and request the decryption of
      $c' = \left\langle c^{'}_{i - 1}, F(c_{i - 1} \oplus m_b)\right\rangle$.

      Then, the decryption oracle gives us the message $m' = F_{k}^{-1}(F_k(c_{i
      - 1} \oplus m_b) \oplus c^{'}_{i - 1}$.

      Due to the definition of CBC, we get the decryption of a message from
      ciphertext $c_i$ by doing $F_{k}^{-1}(c_i) \oplus c_{i - 1}$. Thus,
      plugging in the ciphertext $c'$ we get

      \begin{align*}
        m' &= F_{k}^{-1}(F_k(c_{i - 1} \oplus m_b)) \oplus c^{'}_{i - 1}\\
           &= c_{i - 1} \oplus m_b \oplus c^{'}_{i - 1}\\
           &= c_{i - 1} \oplus m_b \oplus c^{'}_{i - 1} \oplus c_{i - 1}\\
           &= m_b \oplus c^{'}_{i - 1}\\
           &= m_b \oplus c^{'}_{i - 1} \oplus c_{i - 1}\\
           &= m_b \oplus \left\{0, 1\right\}0^{n - 1}
      \end{align*}

      Then, output the bit $b' = 0$ if the string $m_b$ is $10^{n - 1}$, or $b'
      = 1$ if $01^{n - 1}$.

      Therefore, the CBC mode of security is not CCA-secure.
    \end{proof}
\end{enumerate}

\end{document}
