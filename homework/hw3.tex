\documentclass{article}

\usepackage{amsmath}
\usepackage{amsfonts}
\usepackage{amsthm}
\usepackage[a4paper, total={6in, 8in}]{geometry}

\DeclareMathOperator{\Mac}{\textsf{Mac}}
\DeclareMathOperator{\Vrfy}{\textsf{Vrfy}}
\DeclareMathOperator{\negl}{\textsf{negl}}

\title{Homework 3}
\date{23 March 2023}
\author{Nicholas Novak (), Yash Karandikar (),\\
Elise Corwin (), Jadriel Delim ()}

\begin{document}
\maketitle

\begin{enumerate}
  \item Let $F$ be a pseudorandom function. Show that the following MAC for
    messages of length $2n$ is insecure: The shared key is a random $k \in
    \left\{0, 1\right\}^n$. To authenticate a message $m_1 \Vert m_2$ with
    $|m_1| = |m_2| = n$, compute the tag $\left\langle F_k(m_1),
    F_k(F_k(m_2))\right\rangle$.

    \begin{proof}
      Take two messages $m_1, m_2 \in \left\{0, 1\right\}^n$.

      With access to the MAC oracle $\Mac_k$, get the tags of the two messages
      in the following format:

      \begin{align*}
        \Mac_k(m_1 \Vert m_1) &= \left\langle F_k(t_1),
        F_k(F_k(t_1))\right\rangle = \left\langle t_1, t_{1}'\right\rangle\\
        \Mac_k(m_2 \Vert m_2) &= \left\langle F_k(t_2),
        F_k(F_k(t_2))\right\rangle = \left\langle t_2, t_{2}'\right\rangle
      \end{align*}

      From there, the attacker can generate a new message $m_1 \Vert m_2$ with 
      the valid tag $\left\langle t_1, t_{2}'\right\rangle$. And, since we can
      forge a new message that has not been previously asked to the oracle, the 
      MAC is not secure.
    \end{proof}

  \item We explore what happens when the basic CBC-MAC construction is used with
    messages of different lengths.
    \begin{enumerate}
      \item Say the sender and receiver do not agree on the message length in
        advance (and so $\Vrfy_k(m, t) = 1$ iff $t = \Mac_k(m)$, regardless of
        the length of $m$), but the sender is careful to only authenticate
        messages of length $2n$. Show that an adversary can forge a valid tag on
        a message of length $4n$.

        \begin{proof}
          The sender fixes some message $m \in \left\{0, 1\right\}^{2n}$. Then,
          the message is split into two blocks of length $n$, i.e. $m = b_1, b_2$
          and $|b_1| = |b_2| = n$.

          Using CBC-MAC, the sender calculates a tag $t$ for the message $m$:

          \[  
            t = F_k(F_k(0^n \oplus b_1) \oplus b_2)
          \]

          Then, the attacker creates a message $m'$ from the blocks of $m$ so
          that

          \[
            m' = b_1, b_2, t \oplus b_1, b_2
          \]

          With this, the tag for the message $m'$ is calculated as the
          following:


          \begin{align*}
            t' &= F_k(F_k(F_k(F_k(0^n \oplus b_1) \oplus b_2) \oplus (t \oplus
            b_1) \oplus b_2)\\
            &= F_k(F_k(t \oplus (t \oplus b_1) \oplus b_2)\\
            &= F_k(F_k(0 \oplus b_1) \oplus b_2)\\
            &= F_k(F_k(b_1) \oplus b_2)\\
            t' &= t
          \end{align*}

          Thus, we find a different message $m'$ that has the same tag $t$, and
          we have successfully forged a previously unknown tag.
        \end{proof}

      \item Say the receiver only accepts 3-block message (so $\Vrfy_k(m, t) =
        1$ only if $m$ has length $3n$ and $t = \Mac_k(m)$), but the sender
        authenticates messages of any length a multiple of $n$. Show that an
        adversary can forge a valid tag on a new message.

        \begin{proof}
          Fix a message $m \in \left\{0, 1\right\}^{2n}$. Split $m$ into $2$
          blocks of length $n$, such that $m = b_1, b_2$ and $|b_1| = |b_2| =
          n$.

          Then, the tag $t$ with CBC-MAC is calculated as:
          \[
            t = F_k(F_k(0^n \oplus b_1) \oplus b_2)
          \]

          The attacker chooses adds an additional block, $r$, of length $n$, and
          requests the tag of the block $r$ such that $t_r = F_k(0^n \oplus r)$.

          From this, the attacker constructs a new message $m'$ such that:

          \[
            m' = r \Vert (b_1 \oplus t_r) \Vert b_2
          \]

          With this, the tag for the new message $m'$ is calculated as:

          \begin{align*}
            t' &= F_k(F_k(F_k(0^n \oplus r) \oplus (b_1 \oplus t_r)) \oplus b_2)\\
               &= F_k(F_k(t_r \oplus (b_1 \oplus t_r)) \oplus r_2)\\
               &= F_k(F_k(0^n \oplus b_1) \oplus b_2)\\
            t' &= t
          \end{align*}

          Thus, since we have not explicitly asked the $\Mac$ oracle for the tag
          of $m'$, we find that we have successfully forged a tag for a message
          of length $3n$ from a message of length $2n$. Thus, we have shown that
          this attack is possible.
        \end{proof}
    \end{enumerate}
  \item Let $H$ be a collision-resistant hash function. Is $\hat{H}$ defined by
    $\hat{H} = H(H(x))$ necessarily collision resistant?

    \begin{proof}
      Let $x_1, x_2 \in \left\{0, 1\right\}^{*}$ be two arbitrary strings input
      to the hash function $\hat{H}$.

      If we have a collision in $\hat{H}$, i.e. $\hat{H}(x_1) = \hat{H}(x_2)$,
      there is one of two cases where we could have gotten here:
      \begin{enumerate}
        \item If $x_1 \neq x_2$, and we have a collision, that means that in the
          hash function $H$, $H(x_1) = H(x_2)$, and if we rewrite the equation, 
          we find by extension $H(H(x_1)) = H(H(x_2))$.
        \item Otherwise, if $x_1 = x_2$, we find that the hash function $H$
          equals itself, and we can find also that $H(H(x_1)) = H(H(x_2))$.
      \end{enumerate}

      We can see that a collision in $\hat{H}$ only happens if there is a
      collision in $H$ as well. And by assumption that $H$ is collision
      resistant, this only occurs with negligible probability, and thus we find 
      that $\hat{H}$ is collision-resistant as well.
    \end{proof}

  \item Let $h$ be a fixed length hash function for inputs of length $2n$ and
    with output length $n$. Construct hash function $H$ as follows: on a string
    input $x \in \left\{0, 1\right\}^*$ of length $L < 2^n$, do the following:
    \begin{enumerate}
      \item[(i)] Set $B := \left\lceil\frac{L}{n}\right\rceil$ (i.e. the number of
        blocks in $x$). Pad $x$ with zeros so its length is a multiple of $n$.
        Parse the padded result as the sequence of $n$-bit blocks $x_1, \dots,
        x_B$. Set $x_{B+1} := L$, where $L$ is encoded as an $n$-bit string.
      \item[(ii)] Set $z_0 := 0^n$. (This is also called the $IV$.)
      \item[(iii)] For $i = 1, \dots, B + 1$, computer $z_i := h(z_{i - 1} \Vert x_i)$.
      \item[(iv)] Output $z_{B + 1}$.
    \end{enumerate}

    The above construction is called the Merkle-Damg\aa rd transform, and is
    often used to construct a hash function handling arbitrary-length inputs
    given a hash function handling fixed-length inputs. It can be proved that
    if $h$ is collision-resistant, then so is $H$.

    For each of the following modifications to the Merkle-Damg\aa rd transform,
    determine whether the result is collision-resistant or not. If yes, provide
    a proof; if not, demonstrate an attack.

    \begin{enumerate}
      \item Modify the construction so that the input length is not included at
        all (i.e., output $z_B$ and not $z_{B + 1} = h(z_B \Vert L)$).

        \begin{proof}
          Let the block length be some $n$. Fix two strings $x_1, x_2$ such
          that $|x_1| < n$, $|x_2| < n$, and:
          \begin{enumerate}
            \item $x_1 = \left\{0, 1\right\} 0$
            \item $x_2 = \left\{0, 1\right\}$
          \end{enumerate}

          i.e. one of the messages has a zero at the end of it.

          However, since we have no record of length, and we pad all the
          strings with zeroes, once they are padded, we don't know them apart,
          and we have a collision in $H$, i.e $H(x_1) = H(x_2)$. Thus, this 
          variant is \textit{not} collision-resistant.
        \end{proof}

      \item Modify the construction so that instead of outputting $z = h(z_B
        \Vert L)$, the algorithm outputs $z_B \Vert L$.

        \begin{proof}
          Fix two arbitrary-length strings $x_1, x_2$ of length $L_1, L_2$ that 
          cause a collision in $H$, i.e. $H(x_1) = H(x_2)$.

          There are several cases:
          \begin{enumerate}
            \item If $L_1 \neq L_2$, since each part of the hash is independent
              (concatenated together), the strings need to be of same length to
              collide.
            \item If $L_1 = L_2$, with similar reasoning to the case above, the
              only way in which the hashes $z_B$ could be the same for inputs
              $x_1 \neq x_2$, is if there is a collision in $h$.
          \end{enumerate}

          Since we find a collision in the modified scheme if $h$ has a
          collision, and none otherwise, we can determine that the new scheme is
          also collision-resistant.
        \end{proof}

      \item Instead of using an $IV$, just start the computation from $x_1$.
        That is, define $z_1 := x_1$ and then compute $z_i := h(z_{i - 1} \Vert
        x_i)$ for $i = 2, \dots, B + 1$ and output $z_{B + 1}$ as before.

        \begin{proof}
          In the modified scheme, the new hash function is calculated as:

          \[
            h(\dots h(h(h(x_1 \Vert x_2) \Vert x_2) \dots )\Vert L)
          \]

          Since the two first strings are concatenated, each of the two sides
          are independent. Thus, the only way that the hash function will output
          the same value is if $x_1 = x_2$, which does not depend on the order.
          Since we cannot re-order these two values to get the same value,
          except for a collision in $h$ itself, and using the general proof for
          the Merkle-Damg\aa rd transform, we find that this construction is
          also secure.
        \end{proof}

      \item Instead of using a fixed $IV$. set $z_0 := L$ and then compute $z_i
        := h(z_{i - 1} \Vert x_i)$ for $i = 1, \dots, B$ and output $z_B$.

        \begin{proof}
          Assume two arbitrary-length strings $x_1, x_2$ of lengths $L_1$ and
          $L_2$ respectively, where there is a collision $H(x_1) = H(x_2)$.

          Due to the modification, we have several cases:
          \begin{enumerate}
            \item If the lengths of the strings are the same, i.e $L_1 = L_2$, 
              since the first $n$ bits encode the length, the only way that
              there could be a collision is in the hash function $h$.
            \item If the lengths are different, and $L_1 \neq L_2$, the first
              $n$ bits of each hash function are different, and therefore the
              only way that a collision would happen is if the first blocks are
              different and we have a collision in $h$.
          \end{enumerate}

          Thus, we find that this variation is also secure.
        \end{proof}

    \end{enumerate}
\end{enumerate}

\end{document}
