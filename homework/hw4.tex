\documentclass{article}

\usepackage{amsmath}
\usepackage{amsfonts}
\usepackage{amsthm}
\usepackage[a4paper, total={6in, 8in}]{geometry}

\title{Homework 4}
\date{}
\author{Nicholas Novak, Yash Karandikar, Jadriel Delim, Elise Corwin}

\begin{document}
\maketitle

\begin{enumerate}
  \item Compute the following by hand.
    \begin{enumerate}
      \item The final two (decimal) digits of $3^{1000}$.

        To find the last two digits, we can find the number mod 100, i.e.
        $3^{1000} \mod 100$

        \[
          N = 1000 = pq
        \]

        Due to Euler's totient:
        \[
          \phi(100) = 100\left(1 - \frac{1}{2}\right)\left(1 - \frac{1}{5}\right) = 40
        \]

        Then, we can get:

        \begin{align*}
          3 \in \mathbb{Z}^{*}_{100} &\to 3\phi(100) = 1 \pmod{100}\\
                3^{40} &= 1 \pmod{100}\\
          3^{40 * 25} &= 1^{25} \equiv 1 \pmod{100} = \boxed{01}
        \end{align*}

      \item $[101^{4,800,000,023} \mod 35]$.

        \begin{align*}
          [101^{4,800,000,023} \mod 35] &=\\
          [101 \mod 5][101 \mod 7] &=\\
          (1, [3 \mod 7])
        \end{align*}

        \begin{align*}
          (1, [3 \mod 7])^{4,800,000,023} &\equiv (1, [3 \mod 7])\\
                                          &\equiv (1, [3 \mod 7])^{[23 \mod
                                          6]}\\
                                          &\equiv (1, [3 \mod 7])^5\\
                                          &\equiv (1^5, [3^5 \mod 7])\\
                                          &\equiv (1, 5)
        \end{align*}

        \begin{align*}
          (1, 5) &\equiv 1(1, 0) + 5(0, 1)\\
                 &\equiv [1 * 1p + 5 * 1q \mod 35]\\
                 &\equiv [5 + 35 \mod 35]\\
                 &\equiv \boxed{5}
        \end{align*}

    \end{enumerate}
  \item Let $p, N$ be integers with $p | N$ . Prove that for any integer $X$,
    \[
      [[X \mod N ] \mod p ] = [X \mod p ].
    \]

    \begin{proof}
      Let $p, N \in \mathbb{Z}$ with $p \mid N$. By assumption $X \in \mathbb{Z}$.

      Consider $\left[\left[X \mod N\right] \mod p\right]$.

      Let $b = \left[X \mod N\right]$. Let $X = aN + b$ for any $a \in
      \mathbb{Z}$.

      Then, $\left[\left[X \mod N\right] \mod p\right] = \left[b \mod p\right]$.

      Then, consider $\left[X \mod p\right]$.

      Now, $\left[X \mod p\right] = \left[(aN + b) \mod p\right]$.

      Since $p \mid N, p \mid aN$ for some $a \in \mathbb{Z}$.

      So, $\left[x \mod p\right] = \left[(aN + b) \mod p\right] = \left[b \mod
      p\right]$.

      Since $\left[\left[X \mod N\right] \mod p\right] = \left[b \mod p\right]$
      and $\left[X \mod p\right] = \left[b \mod p\right]$, $\left[\left[X \mod
      N\right] \mod p\right] = \left[X \mod p\right]$
    \end{proof}

    Show that, in contrast, $[[X \mod p ] \mod N ]$ need not equal $[X \mod N]$.

    \begin{proof}
      Suppose towards contradiction that $\forall p, N \in \mathbb{Z}$
      $\left[\left[X \mod p\right] \mod N\right] = \left[X \mod N\right]$.

      Let $b = \left[X \mod p\right]$. Let $X = bN + a$.

      Then $\left[\left[X \mod p\right] \mod N\right] = b \mod N$, but $\left[X
      \mod N\right] = \left[(bN + a) \mod N\right]$, and since $N \mid bN$ for
      some $b \in \mathbb{Z}$, $\left[(bN + a) \mod N\right] = \left[a \mod
      N\right]$.

      So, for some value of $a, b \in \mathbb{Z}$, if $x = bN + a, \left[\left[X
      \mod p\right] \mod N\right] \neq \left[X \mod N\right]$.

      Therefore, we can see that no such $X$ exists.
    \end{proof}

  \item Show the following.
    \begin{enumerate}
      \item If $N = pq$ and $ed = 1 \mod \phi(N)$ then for all $x \in Z^{*}_N$ 
        we have $(x^e)^d = x \mod N$

        \begin{proof}
          Let $p, q \in \mathbb{Z}$ such that $p, q$ are distinct primes. $N =
          pq$. Let $e, d \in \mathbb{Z}$ such that $ed \equiv 1 \pmod{\phi(N)}$.
          Consider $\phi(N)$.

          Since $N$ is composite, $\phi(N) = (p - 1)(q - 1)$.

          Since $ed \equiv 1 \pmod{\phi(N)}$, $ed \equiv \pmod{(p - 1)(q - 1)}$,
          so $ed \equiv 1 + (p - 1)(q - 1)k$, for some $k \in \mathbb{Z}$.

          Now consider, $x^{ed}$. $x^{ed} = x^{(p - 1)(q - 1) + 1} = x^{(p -
            1)}x^{(q - 1)}x$

          Now, since $p - 1$ and $q - 1$ are inverses ($\mod (p - 1)(q - 1)$),
          $x^{p - 1}x^{q - 1}x = 1 * 1 * x = x$.

          So, $x^{ed} = x$.
         \end{proof}

      \item If $N = pq$ and $ed = 1 \mod \phi(N)$ then for all $x \in Z_N$ we
        have $(x^e)^d = x \mod N$.
    \end{enumerate}
  \item 
    \begin{enumerate}
      \item Formally define the CDH assumption

      Given g, $h_1, h_2$ compute $DH_g(h_1, h_2)$ where $h_1$ and $h_2$ are given randomly chosen $h_1$ and $h_2$ and g is a generator that is a cyclic group G. 
      If the discrete log problem in G is easy then the CDH problem is also easy. 
      Given $h_1$ and $h_2$:
      1. Compute DH($h_1$, $h_2$)
      
      \item Prove that hardness of the CDH problem relative to $\mathcal{G}$ implies
        hardness of the discrete-logarithm problem relative to $\mathcal{G}$.
      \item Prove that hardness of the DDH problem relative to $\mathcal{G}$
        implies hardness of the CDH problem relative to $\mathcal{G}$.
    \end{enumerate}
  \item Let \textsf{GenRSA} be a probabilistic polynomial-time algorithm that,
    on input $1^n$, outputs a modulus $N$ that is the product of two $n$-bit 
    primes, as well as integers $e, d > 0$ with $\gcd(e, \phi(N)) = 1$ and 
    $ed = 1 \mod \phi(N)$.

    Define $(\textsf{Gen}, H)$ as follows:

    \begin{enumerate}
      \item \textsf{Gen}: on input $1^n$, run $\textsf{GenRSA}(1^n)$ to obtain
        $N, e, d$, and select $y \leftarrow \mathbb{Z}^{*}_N$. The key is $s := \left<N, e,
        y\right>$.

        \begin{proof}
          Suppose the RSA problem is hard relative to \textsf{GenRSA}. Then
          \textsf{GenRSA} cannot solve \textsf{RSA} in polynomial time. So, it
          \textsf{GenRSA} is restricted to polynomial time, it cannot solve
          \textsf{RSA}. This implies that the probability of the experiment
          succeeding is negligible.
        \end{proof}

      \item \textsf{H}: if $s = \left<N, e, y\right>$ , then $H^s$ maps inputs
        in $\left\{0, 1\right\}^{3n}$ to outputs in $\mathbb{Z}^{*}_N$. Let $f_0^s(x) :=
        [x^e \mod N]$ and $f_1^s(x) := [y \cdot x^e \mod N]$. For a $3n$-bit long
        string $x = x_1 \dots x_{3n}$, define

        \[
          H^s(x) := f_{x_1}^s(f_{x_2}^s(\dots(1)\dots)).
        \]

        Prove that if the RSA problem is hard relative to \textsf{GenRSA} than
        $H$ is a fixed-length collision-resistant hash function.

        \begin{proof}
          So, \textsf{GenRSA} provides $N, e, d$ and \textsf{Gen}, as well as $y
          \leftarrow \mathbb{Z}^{*}_{N}$.

          This is used by $H^S$ to map $\left\{0, 1\right\}^{3n}$ to
          $\mathbb{Z}^{*}_{N}$. For a collision to occur, two distinct inputs
          must map to the same output. This occurs only when the identity
          element in the group is provided.

          Since $\mathbb{Z}^{*}_N$ has a unique identity there is no collision.
        \end{proof}

    \end{enumerate}
\end{enumerate}

\end{document}